
%\section{安装}
\chapter{安装}


Gate是一个开源的项目,其源代码公布在github网站上(https://github.com/OpenGATE/Gate),
可以通过Git版本控制软件下载完整的程序代码库,

\begin{lstlisting}
git clone https://github.com/OpenGATE/Gate.git
\end{lstlisting}


也可以下载最新的发布版,然后解压、编译。目前,最新的版本是 V-8.0。

每个人都可以通过这个网站克隆代码库,然后添加、修改代码或者修复bugs,为项目做贡献。

Gate软件运行系统推荐用Linux系统,如Centos、Ubuntu、Archlinux、Debian等Linux发行版。

在安装Gate之前,除了必备的编译环境(gcc、make、OpenGL等)外,这几个软件包必需事先装好,分别是:

\begin{itemize}
	\item CMake 用于配置编译环境,生成make所需要的Makefile。这个工具在网站https://cmake.org/下载对应系统的最新二进制版本,解压即可使用
	\item Geant4 Gate基于Geat4基础上开发,通过网址http://geant4.cern.ch/support/download.shtml下载Geant4的相关代码和核数据库。
	\item ROOT  Gate模拟的数据输出可提供多种格式,如ASSIC、Binary、ROOT、Interfile、Sinogram、LMF等。
			主要推荐ROOT格式,配合ROOT数据分析软件,实现数据分析和图像呈现。可在root.cern.ch网站上下载安装。
\end{itemize}

注意:目前,Gate的发布版本还未支持Geant4的多线程(MT mode),因此,安装Geant4时注意不要开启多线程模式。

\subsection{cmake的安装}
从cmake的官网上下载linux版本二进制安装包,如图\ref{fig:cmake-site}

\begin{figure}[H]
	\centering
	\includegraphics[width=0.7\linewidth]{image/cmake-site}
	\caption[]{cmake官网下载示意图}
	\label{fig:cmake-site}
\end{figure}
在PATH系统环境变量中加入cmake的执行命令路径。假设cmake安装的目录为~/opt/cmake,在bash命令行终端中输入:

export PATH=~/opt/cmake/bin:\$PATH  %配置环境变量PATH

如果是csh终端,则用

setenv PATH=~/opt/cmake/bin:\$PATH


\subsection{ROOT的安装}
ROOT的代码可以从ROOT的官网(root.cern.ch)下载,亦可通过git在https://github.com/root-project/root/网上克隆对应的版本进行安装。
比如下载root\_v6.12.06.source.tar.gz代码到~/opt/目录下,相应的命令行操作如下:

cd ~/opt

tar -zxvf root\_v6.12.06.source.tar.gz  %解压源代码

mkdir root-build

cd root-install

ccmake ..//root-6.12.06  %ccmake配置root的编译环境

make  %编译代码

make install   %安装ROOT软件到指定的安装目录下

cmake配置root源代码编译环境环境界面如图\ref{fig:root-install-configure},其中CMAKE\_INSTALL\_PREFIX选项指定安装路径。
本例子安装到~/opt/root\_6.12.06\_install
\begin{figure}[H]
	\centering
	\includegraphics[width=0.7\linewidth]{image/root-install-configure}
	\caption{CMake配置ROOT的源代码编译环境}
	\label{fig:root-install-configure}
\end{figure}



\subsection{Geant4的安装}
Geant4的代码和所需要的数据库可以从Geant4的官网(geant4.cern.ch)下载,如图\ref{fig:geant4-download}。Geant4一般每半年更新一次,目前最新版本为geant4-10.4.1。

\begin{figure}[H]
	\centering
	\includegraphics[width=0.7\linewidth]{image/geant4-download}
	\caption{geant4官网的下载网页界面}
	\label{fig:geant4-download}
\end{figure}


本例子假设如下载geant4.10.04.p01.tar.gz代码到~/opt/目录下,相应的命令行操作如下:

cd ~/opt

tar -zxvf geant4.10.04.p01.tar.gz  %解压源代码

mkdir geant4-build

cd geant4-build

ccmake ../geant4.10.04.p01  %ccmake配置root的编译环境

make  %编译代码

make install   %安装ROOT软件到指定的安装目录下

cmake配置geant4源代码编译环境环境界面如图\ref{fig:geant4-install-configure},其中CMAKE\_INSTALL\_PREFIX选项指定安装路径。
本例子安装到~/opt/geant4.10.4.p1\_install。geant4支持Qt可视化,可选上GEANT4\_USE\_QT,具体参考图\ref{fig:geant4-install-configure}配置

\begin{figure}[H]
	\centering
	\includegraphics[width=0.7\linewidth]{image/geant4-install-configure}
	\caption{CMake配置Geant4的源代码编译环境}
	\label{fig:geant4-install-configure}
\end{figure}

如果系统安装的Qt是自己安装的,需要修改源代码下的CMakeLists.txt文件,其中加入Qt的安装路径,如图\ref{fig:addqtingeant4}
\begin{figure}[h]
	\centering
	\includegraphics[width=0.7\linewidth]{image/add_qt_in_geant4}
	\caption{Geant4手动增加Qt的编译环境}
	\label{fig:addqtingeant4}
\end{figure}


\subsection{Gate的安装}
Gate的代码可以通过Gate官网(http://www.opengatecollaboration.org)下载,也能在github(https://github.com/OpenGATE/Gate)网站上下载.
目前最新版本Gate\_v8.0。

安装完Geant4和ROOT这两个软件后,需要配置一下各自的运行环境,即设置系统的环境变量。
对于Geant4,需要运行一下source ~/opt/geant4/geant4.10.4.p1\_install/bin/geant4.sh,如果是csh,则是geant4.csh。
而ROOT的运行环境也有一个脚本实现配置,source ~/opt/root\_6.12.06\_install/bin/thisroot.sh。

cd ~/opt

tar -zxvf gate\_v8.0.tar.gz  %解压源代码

mkdir gate-build

cd gate-build

ccmake ../gate\_v8.0  %ccmake配置Gate的编译环境

make  %编译代码

make install   %安装Gate软件到指定的安装目录下

至此,Gate程序安装完毕。

可以在命令行中运行:
./Gate --qt

如果出现如图\ref{fig:gate-qt-gui}的界面,表明Gate运行正常。
\begin{figure}[H]
	\centering
	\includegraphics[width=0.7\linewidth]{image/Gate-qt-gui}
	\caption{Gate调用Qt界面}
	\label{fig:gate-qt-gui}
\end{figure}
