
%\section{简介}
\chapter{简介}

Gate一个用于核影像物理的蒙卡模拟程序,包括X射线断层扫描、正电子湮灭断层扫描、射线治疗等核相关模拟。

这个程序基于欧洲核子中心开发的Geant4程序库开发而来应用程序,使用C++编程语言。
Gate的输入文件基于Geant4的交互命令扩展而来,不需要用到C++程序语言实现Gate的模拟功能,
因此,安装后Gate程序后,直接使用Gate的特有命令语句,即可构建模型,输出数据。

Gate使用一个.mac结尾的脚本作为输入文件,这个脚本定义了蒙卡模拟相关的几个主要部分,分别为:
\begin{itemize}
	\item scanner  探测器部件,定义探测器的几何形状、位置、材料,运动状态;
	\item phantom  模体,定义模体的几何形状、位置、材料,运动状态;
	\item physic  物理过程,定义射线与物质的物理作用过程;
	\item digiter  数字化过程,定义信号输出参数,(能量收集和输出分布,信号时间特性(如延迟、时间窗、死时间等));
	\item source  放射源,定义放射源的几何形状,位置,运动状态,射线类别、能谱、发射分布,放射性强度等;
	\item output 数据输出控制,选择保存数据的格式及数据信息;
	\item 模拟运行时间或粒子数目;
	\item 开始运行指令(/gate/application/startDAQ)。
\end{itemize}

\section{成像模拟的基本框架}
成像模拟的基本框架主要包括如下步骤:
\begin{itemize}
	\item 定义扫描头几何结构;
	\item 定义模体几何结构;
	\item 定义物理过程;
	\item 初始化模拟(/gate/run/initialize)
	\item 设置探测器工作模型
	\item 设置放射源
	\item 指定数据输出格式
	\item 开始采集
\end{itemize}
其中,步骤一到四为模拟的初始化,必须放在/gate/run/initialize语句之前,初始化后,几何结构不可更改。

\section{放疗及剂量模拟的基本框架}
放疗及剂量模拟的基本框架主要包括如下步骤:
\begin{itemize}
	\item 定义粒子速流几何;
	\item 定义模体几何;
	\item 指定输出数据形式;
	\item 设置物理过程;
	\item 初始化模拟(/gate/run/initialize);
	\item 定义放射源;
	\item 开始模拟。
\end{itemize}

举个栗子:

\begin{lstlisting}
is a code!
\end{lstlisting}

